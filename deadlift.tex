\documentclass{article}
\usepackage{draftwatermark}
\SetWatermarkText{Draft}
\SetWatermarkScale{5}

\begin{document}
\newcommand{\emojiname}{`person deadlifting'}
\begin{abstract}
The intention of this proposal is to advocate for the inclusion of a new emoji, \emojiname{} , as a Unicode emoji character. 
Amongst professional Power Lifters, professional Olympic lifters, and amature lifters the deadlift is a core training lift. 
For Power Lifters, it is one of the three lifts in competition. For Olympic lifters, it is the first phase of all lifts recognised by the International Olympic Commitee
\end{abstract}
\section{Identification}
The Deadlift Emoji

\section{Image}
The image of the deadlift emoji

\section{Selection factors}
\subsection{Compatibility}
At the present time, no system currently has an emoji representing a person performing the deadlift. The `person lifting weights' emoji is a related emoji that
exclussively and in almost all platforms represents a clean and jerk \footnote{https://unicode.org/emoji/charts/full-emoji-list.html numbers 1154-1171}.
However, a quick search on Twitter for the hashtag `deadlift' will reveal a steady flow of users on social media using the `person lifting weights' emoji in reference to deadlifts .

\subsection{Expected usage level}
\subsection{Frequency}
The expected usuage of the proposed emoji would be widespread. Amongst Olympic lifters, Power Lifters, and those who may have an interest in general health and conditioning but may not even do the
lift.

\subsection{Image distinctivenes}
\subsection{Completeness}
\subsection{Frequently requested}

\section{Counterarguments to Factors for Exclusion}
\subsection{Overly specific}
While the propsed emoji \emojiname{} is a specific lift in and of itself. The emoji can be used to denote general fitness and other Olympic lifts. It is specific enough to denote 
a particular lift, but generic enough to denote general forms of exercise, whether they be resistence training or otherwise. A specific lifting emoji would be a curl, that is often
used for a very particular purpose. However, a variety of people may use the deadlift emoji due to its international acceptance by the International Weightlifting Federation, and International Powerlifting Federation\footnote{http://www.iwf.net/2009/07/26/world-records-broken/

http://www.powerlifting-ipf.com/about-ipf/disciplines.html
}

\subsection{Open-ended}
\subsection{Already representable}
Thhere is no emoji that prepresents \emojiname{}.

\subsection{Logos, brands, UI icons, signage, specific people, deities}
\subsection{Transient}
\subsection{Faulty comparison}
\subsection{Exact Images}

\section{Sort location}

\end{document}