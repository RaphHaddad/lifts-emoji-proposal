\documentclass{article}

\usepackage{draftwatermark}
\SetWatermarkText{Draft}
\SetWatermarkScale{5}

\begin{document}

\begin{abstract}
The intention of this proposal is to advocate for the inclusion of a new emoji, the deadlift, as a Unicode emoji character. 
Amongst professional Power Lifters, professional Olympic lifters, and amature lifters the deadlift is a core training lift. 
For Power Lifters, it is one of the three lifts in competition. For Olympic lifters, it is the first phase of all lifts recognised by the International Olympic Commitee
\end{abstract}
\section{Identification}
The Deadlift Emoji

\section{Image}
The image of the deadlift emoji

\section{Selection factors}
\subsection{Compatibility}
At the present time, no system currently has an emoji representing a person performing the deadlift. 

\subsection{Expected usage level}
The expected usuage of this emoji would be widespread. Amongst Olympic lifters, Power Lifters, and those who may have an interest in general health and conditioning but may not even do the
lift.

\subsection{Image distinctivenes}
\subsection{Completeness}
\subsection{Frequently requested}

\section{Counterarguments to Factors for Exclusion}
\subsection{Overly specific}
\subsection{Open-ended}
\subsection{Already representable}
\subsection{Logos, brands, UI icons, signage, specific people, deities}
\subsection{Transient}
\subsection{Faulty comparison}
\subsection{Exact Images}

\section{Sort location}

\end{document}